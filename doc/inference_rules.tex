\documentclass{article}

\usepackage[margin=2.5cm]{geometry}
\usepackage{amsmath}
\usepackage{amsfonts}
\usepackage{amssymb}
\usepackage{bussproofs}
\usepackage{xcolor}
\usepackage{graphicx}

\newcommand{\op}[3]{[#1] \texttt{#2} [#3]}

\newenvironment{scprooftree}[1]%
{\gdef\scalefactor{#1}
\begin{center}\proofSkipAmount \leavevmode}%
  {\scalebox{\scalefactor}{\DisplayProof}\proofSkipAmount
\end{center} }

\title{IV Type System Inference Rules}

\begin{document}

\maketitle

\section*{Language}

\subsection*{Declarations}

\subsubsection*{Data declaration}
The data types are represented by Algebraic Data Types (ADT) in the language.
\begin{verbatim}
data Either a b:
  [a] left,
  [b] right.
\end{verbatim}
If a constructor does not take any types its input parameters can be omitted.
\begin{verbatim}
data Maybe a:
  nothing,
  [a] just.
\end{verbatim}
ADT's can be recursive.
\begin{verbatim}
data Nat:
  zero,
  [Nat] suc.
\end{verbatim}

\subsection*{Operator declaration}
The operators are represented by a sequence of operators. All operator
definitions must have a type annotation.
\begin{verbatim}
define [Nat, Nat] add [Nat]:
  case { zero { }, suc { add suc } }.

define [Nat] addThree [Nat]:
  zero suc suc suc add.
\end{verbatim}
An operator's body can be empty
\begin{verbatim}
define [] nop []:.
\end{verbatim}

\subsection*{Standard operations}

\subsubsection*{Non-parametric}
\begin{itemize}
  \item \texttt{pop} \\
    Delete the top element.
  \item \texttt{dup} \\
    Duplicate the top element.
\end{itemize}

\subsubsection*{Parametric}
Parameter \texttt{n} is a natural number without zero ($\mathbb{N}
\setminus \{ 0 \}$).
\begin{itemize}
  \item \texttt{br-n} \\
    Move the topmost element to be the n-th element.
  \item \texttt{dg-n} \\
    Move the n-th element to the top of the stack.
\end{itemize}

\section*{Operator type system}

\subsection*{Environment}
Environment is constant throughout the type checking process. It
contains operator definitions and data definitions.
\begin{equation*}
  \Gamma = (\text{opDefs}, \text{dataDefs})
\end{equation*}
Operator definitions consist of standard operators, user-defined
operators, user-defined data constructors.
\begin{equation*}
  \text{opDefs} = \text{stdOps} \cup \text{userOps} \cup \text{userDataConstrs}
\end{equation*}
Data definitions consist of user-defined data types (Algebraic Data Types).
\begin{equation*}
  \text{dataDefs} = \text{userDatas}
\end{equation*}

\subsection*{Operator Type separation}
\begin{itemize}
  \item \textbf{Type} - represents the type of a value stored on the stack.
  \item \textbf{Operator Type} - represents the type of an element of
    an operator body.
\end{itemize}

\subsection*{Type definition}
A type can be one of the following
\begin{itemize}
  \item Monomorphic type, e.g., Int, Bool, Nat.
  \item Polymorphic type (type variable), e.g., a, b, c.
  \item Type application, e.g. Maybe a, Either a b.
\end{itemize}
Definition in Haskell
\begin{verbatim}
data Type
  = Mono String
  | Poly String
  | App Type Type
\end{verbatim}

\subsection*{Operator Type definition}
An operator has only one constructor that has the following fields
\begin{itemize}
  \item pre - types of elements that the operator takes as input arguments.
  \item post - types of elements that the operator returns as output arguments.
\end{itemize}
Definition in Haskell
\begin{verbatim}
data OpType
  = OpType {
    pre :: [Type],
    post :: [Type]
  }
\end{verbatim}

\subsection*{Notation}
\begin{itemize}
  \item ``$\op{\textit{pre}}{}{\textit{post}}$'' represents an
    operator type. Where \textit{pre} and \textit{post} represent the
    input and output parameters.
  \item ``$\{ \text{a} \mapsto \text{Foo} \}
    \op{\textit{pre}}{}{\textit{post}}$'' represents an application
    of a substitution ``$\{ \text{a} \mapsto \text{Foo} \}$'' on an
    operator type ``$\op{\textit{pre}}{}{\textit{post}}$''
  \item ``$\alpha \cdot \beta$'' represents list concatenation.
  \item An operator(s) between the stack descriptions is a shorthand,
    e.g., writing $\op{a, b}{foo bar}{c, d}$ is equivalent to
    $\texttt{foo bar} : \op{a, b}{}{c, d}$.
  \item The leftmost element in the stack type description is the
    most recently pushed,\\
    e.g., $\op{}{foo bar baz}{\text{Baz}, \text{Bar}, \text{Foo}}$.
  \item Greek letters denote lists of types, while Latin letters
    denote single types.
\end{itemize}

\subsection*{Type inference rules}
\subsubsection*{Specialization Rule (Operator Type)}
An operator type is considered a specific of a general operator type
if there exists a substitution that turns the general type into the
specific type.
\begin{prooftree}
  \AxiomC{$\op{\alpha}{}{\beta} = \{ a' \mapsto a \} \op{\alpha'}{}{\beta'}$}
  \UnaryInfC{$\op{\alpha}{}{\beta} \sqsubseteq \op{\alpha'}{}{\beta'}$}
\end{prooftree}

\subsubsection*{Empty rule}
Allows to use an empty sequence of operators, that does not take any
input arguments and does not return any output arguments.
\begin{prooftree}
  \AxiomC{}
  \UnaryInfC{$\Gamma \vdash \op{}{}{}$}
\end{prooftree}

\subsubsection*{Name rule}
Allows to use previously defined operators.
\begin{prooftree}
  \AxiomC{$\op{\alpha}{op}{\beta} \in \Gamma$}
  \UnaryInfC{$\Gamma \vdash \op{\alpha}{op}{\beta}$}
\end{prooftree}

\subsubsection*{Specialization and augmentation rule}
Allows specialization and augmentation of operator types of operators.
\begin{itemize}
  \item (Specialization) allows to use $\op{a}{id}{a}$ in place of
    $\op{\text{Nat}}{inc}{\text{Nat}}$
  \item (Augmentation) allows to use \op{\text{Nat}}{inc}{\text{Nat}}
    in place of \op{\text{Nat}, \text{Nat}}{inc2}{\text{Nat}, \text{Nat}}
\end{itemize}
\begin{prooftree}
  \AxiomC{$\Gamma \vdash \op{\alpha'}{x}{\beta'}$}
  \AxiomC{$\op{\alpha}{}{\beta} \sqsubseteq \op{\alpha'}{}{\beta'}$}
  \BinaryInfC{$\Gamma \vdash \op{\alpha \cdot \gamma}{x}{\beta \cdot \gamma}$}
\end{prooftree}

\subsubsection*{Chain rule}
Allows to compose operators. To be chained, LHS post should be equal
(i.e., equal length and elements, including type variables) to the RHS pre.
\begin{prooftree}
  \AxiomC{$\Gamma \vdash \op{\alpha}{x}{\beta}$}
  \AxiomC{$\Gamma \vdash \op{\psi}{y}{\omega}$}
  \AxiomC{$\beta = \psi$}
  \TrinaryInfC{$\Gamma \vdash \op{\alpha}{x y}{\omega}$}
\end{prooftree}

\subsubsection*{Case rule}
Operator type of the whole case expression must be a specific of all
case arms. Pattern matching should be total on all constructors of
the data type.
\begin{prooftree}
  \AxiomC{$\{ \texttt{constr1}, \ldots \} = \text{constrs} ( t )$}
  \AxiomC{$\Gamma \vdash \op{t, \alpha'}{constr$^{-1}$ body}{\beta'}$}
  \AxiomC{$\op{t, \alpha}{}{\beta} \sqsubseteq \op{t, \alpha'}{}{\beta'}$}
  \AxiomC{$\ldots$}
  \QuaternaryInfC{$\Gamma \vdash \op{t,
  \alpha}{case\{constr1\{body1\},...\}}{\beta}$}
\end{prooftree}
A case arm operator type is destructor $\texttt{constr$^{-1}$}$
chained with the body. \\
Where $\texttt{constr$^{-1}$}$ is the destructor of a constructor
$\texttt{constr}$, i.e., Operator Type of $\texttt{constr}$ with pre
and post swapped.
\begin{prooftree}
  \AxiomC{$\Gamma \vdash \op{\alpha}{constr}{t}$}
  \UnaryInfC{$\Gamma \vdash \op{t}{constr$^{-1}$}{\alpha}$}
\end{prooftree}

\subsubsection*{Stack operations}
Dup
\begin{prooftree}
  \AxiomC{}
  \UnaryInfC{$\Gamma \vdash \op{a}{dup}{a, a}$}
\end{prooftree}
Pop
\begin{prooftree}
  \AxiomC{}
  \UnaryInfC{$\Gamma \vdash \op{a}{pop}{}$}
\end{prooftree}
Bury
\begin{prooftree}
  \AxiomC{$||\alpha|| = n$}
  \UnaryInfC{$\Gamma \vdash \op{b, \alpha}{br-n}{\alpha, b}$}
\end{prooftree}
Dig
\begin{prooftree}
  \AxiomC{$||\alpha|| = n$}
  \UnaryInfC{$\Gamma \vdash \op{\alpha, b}{dg-n}{b, \alpha}$}
\end{prooftree}

\end{document}
